% Options for packages loaded elsewhere
\PassOptionsToPackage{unicode}{hyperref}
\PassOptionsToPackage{hyphens}{url}
\PassOptionsToPackage{dvipsnames,svgnames,x11names}{xcolor}
%
\documentclass[
]{agujournal2019}

\usepackage{amsmath,amssymb}
\usepackage{iftex}
\ifPDFTeX
  \usepackage[T1]{fontenc}
  \usepackage[utf8]{inputenc}
  \usepackage{textcomp} % provide euro and other symbols
\else % if luatex or xetex
  \usepackage{unicode-math}
  \defaultfontfeatures{Scale=MatchLowercase}
  \defaultfontfeatures[\rmfamily]{Ligatures=TeX,Scale=1}
\fi
\usepackage{lmodern}
\ifPDFTeX\else  
    % xetex/luatex font selection
\fi
% Use upquote if available, for straight quotes in verbatim environments
\IfFileExists{upquote.sty}{\usepackage{upquote}}{}
\IfFileExists{microtype.sty}{% use microtype if available
  \usepackage[]{microtype}
  \UseMicrotypeSet[protrusion]{basicmath} % disable protrusion for tt fonts
}{}
\makeatletter
\@ifundefined{KOMAClassName}{% if non-KOMA class
  \IfFileExists{parskip.sty}{%
    \usepackage{parskip}
  }{% else
    \setlength{\parindent}{0pt}
    \setlength{\parskip}{6pt plus 2pt minus 1pt}}
}{% if KOMA class
  \KOMAoptions{parskip=half}}
\makeatother
\usepackage{xcolor}
\setlength{\emergencystretch}{3em} % prevent overfull lines
\setcounter{secnumdepth}{5}
% Make \paragraph and \subparagraph free-standing
\makeatletter
\ifx\paragraph\undefined\else
  \let\oldparagraph\paragraph
  \renewcommand{\paragraph}{
    \@ifstar
      \xxxParagraphStar
      \xxxParagraphNoStar
  }
  \newcommand{\xxxParagraphStar}[1]{\oldparagraph*{#1}\mbox{}}
  \newcommand{\xxxParagraphNoStar}[1]{\oldparagraph{#1}\mbox{}}
\fi
\ifx\subparagraph\undefined\else
  \let\oldsubparagraph\subparagraph
  \renewcommand{\subparagraph}{
    \@ifstar
      \xxxSubParagraphStar
      \xxxSubParagraphNoStar
  }
  \newcommand{\xxxSubParagraphStar}[1]{\oldsubparagraph*{#1}\mbox{}}
  \newcommand{\xxxSubParagraphNoStar}[1]{\oldsubparagraph{#1}\mbox{}}
\fi
\makeatother


\providecommand{\tightlist}{%
  \setlength{\itemsep}{0pt}\setlength{\parskip}{0pt}}\usepackage{longtable,booktabs,array}
\usepackage{calc} % for calculating minipage widths
% Correct order of tables after \paragraph or \subparagraph
\usepackage{etoolbox}
\makeatletter
\patchcmd\longtable{\par}{\if@noskipsec\mbox{}\fi\par}{}{}
\makeatother
% Allow footnotes in longtable head/foot
\IfFileExists{footnotehyper.sty}{\usepackage{footnotehyper}}{\usepackage{footnote}}
\makesavenoteenv{longtable}
\usepackage{graphicx}
\makeatletter
\newsavebox\pandoc@box
\newcommand*\pandocbounded[1]{% scales image to fit in text height/width
  \sbox\pandoc@box{#1}%
  \Gscale@div\@tempa{\textheight}{\dimexpr\ht\pandoc@box+\dp\pandoc@box\relax}%
  \Gscale@div\@tempb{\linewidth}{\wd\pandoc@box}%
  \ifdim\@tempb\p@<\@tempa\p@\let\@tempa\@tempb\fi% select the smaller of both
  \ifdim\@tempa\p@<\p@\scalebox{\@tempa}{\usebox\pandoc@box}%
  \else\usebox{\pandoc@box}%
  \fi%
}
% Set default figure placement to htbp
\def\fps@figure{htbp}
\makeatother

\usepackage{url} %this package should fix any errors with URLs in refs.
\usepackage{lineno}
\usepackage[inline]{trackchanges} %for better track changes. finalnew option will compile document with changes incorporated.
\usepackage{soul}
\linenumbers
\makeatletter
\@ifpackageloaded{caption}{}{\usepackage{caption}}
\AtBeginDocument{%
\ifdefined\contentsname
  \renewcommand*\contentsname{Table of contents}
\else
  \newcommand\contentsname{Table of contents}
\fi
\ifdefined\listfigurename
  \renewcommand*\listfigurename{List of Figures}
\else
  \newcommand\listfigurename{List of Figures}
\fi
\ifdefined\listtablename
  \renewcommand*\listtablename{List of Tables}
\else
  \newcommand\listtablename{List of Tables}
\fi
\ifdefined\figurename
  \renewcommand*\figurename{Figure}
\else
  \newcommand\figurename{Figure}
\fi
\ifdefined\tablename
  \renewcommand*\tablename{Table}
\else
  \newcommand\tablename{Table}
\fi
}
\@ifpackageloaded{float}{}{\usepackage{float}}
\floatstyle{ruled}
\@ifundefined{c@chapter}{\newfloat{codelisting}{h}{lop}}{\newfloat{codelisting}{h}{lop}[chapter]}
\floatname{codelisting}{Listing}
\newcommand*\listoflistings{\listof{codelisting}{List of Listings}}
\makeatother
\makeatletter
\makeatother
\makeatletter
\@ifpackageloaded{caption}{}{\usepackage{caption}}
\@ifpackageloaded{subcaption}{}{\usepackage{subcaption}}
\makeatother

\usepackage{bookmark}

\IfFileExists{xurl.sty}{\usepackage{xurl}}{} % add URL line breaks if available
\urlstyle{same} % disable monospaced font for URLs
\hypersetup{
  pdftitle={Arizona Aquifer Recharge Suitability Analysis},
  pdfauthor={Travis Zalesky},
  pdfkeywords={Arizona Tri University Recharge (ATUR), water
table, ground water},
  colorlinks=true,
  linkcolor={blue},
  filecolor={Maroon},
  citecolor={Blue},
  urlcolor={Blue},
  pdfcreator={LaTeX via pandoc}}



\draftfalse

\begin{document}
\title{Arizona Aquifer Recharge Suitability Analysis}

\authors{Travis Zalesky\affil{1}}
\affiliation{1}{University of Arizona, }
\correspondingauthor{Travis Zalesky}{travisz@arizona.edu}


\begin{abstract}
Aquifer recharge can be either passive or active, and is implemented in
a variety of ways. This analysis seeks to identify regions across AZ
which are boadly suitable for aquifer recharge projects as a general
template for more focuse analysis.
\end{abstract}

\section*{Plain Language Summary}
Identifying regions in AZ where surface water can be stored long-term as
ground water.




\section{Introduction}\label{introduction}

\section{Data \& Methods}\label{sec-data-methods}

\subsection{Elevation}\label{elevation}

\subsubsection{DEM}\label{dem}

Elevation and elevation derivatives from 30-m NASA SRTM. USGS 3-DEM
(10m) product not suitable for full study area analysis due to (1) the
large area of missing data in Mexico, and (2), the excessively high
spatial resolution (massively increasing computational requirements).

SRTM elevation sinks filled prior to calculating slope and aspect.

\textbf{Should elevation be directly used in the suitability analysis?}

\subsubsection{Slope}\label{slope}

Slope derived from hydrologically conditioned (filled) 30-m SRTM layer
using quadratic surface function and a fixed 30-m neighborhood. Slope
measured in °.

\begin{quote}
Higher slopes are less suitable because thinning is both more expensive
and more precipitation will end up as runoff.
\end{quote}

Slope classified from 1-10 using a continuous function in ArcPro
Suitability Mapper.

\subsubsection{Aspect}\label{aspect}

Aspect calculated as with slope. Aspect reference point at N. Pole.

\begin{quote}
Aspect has a large impact on solar radiation. Closer to 0 or 360 is
desired, low suitability scores for closeness.
\end{quote}

Aspect classified from 1-10 using a continuous function in ArcPro
Suitability Mapper.

\subsection{Precipitation}\label{precipitation}

\textbf{Data source?}

\begin{itemize}
\tightlist
\item
  \textbf{PRISM normals}

  \begin{itemize}
  \tightlist
  \item
    800m resolution
  \item
    All months (30-Y)
  \end{itemize}
\item
  \textbf{Custom PRISM}

  \begin{itemize}
  \tightlist
  \item
    1Km resolution
  \item
    Subset months of interest
  \item
    Custom date range
  \item
    Custom averaging function
  \item
    More granular control over data
  \end{itemize}
\end{itemize}

\textbf{Also applies to Temp and other Climactic variables of interest.}

\section{Conclusion}\label{sec-conclusions}

\section*{References}\label{references}
\addcontentsline{toc}{section}{References}




\end{document}
